\chapter{Metodologie agili}
% \label{app:agile}

\section{Definizione}
Le metodologie agili sono dei modelli di ciclo di vita di un \textit{software}, ovvero un insieme di:
\begin{itemize}
    \item \textbf{Stati}: indicazioni di maturità del prodotto, identificano l'avanzamento macroscopico dei lavori e l'insieme di determinate caratteristiche di base degli artefatti prodotti;
    \item \textbf{Transizioni}: cambiamenti di stato di un prodotto spinti da specifiche attività di specifici processi (ad esempio, l'\textbf{analisi dei requisiti} è un'attività, cioè un insieme di compiti, facente parte del processo primario "\textbf{sviluppo}");
    \item \textbf{Processi}: un processo è un insieme di attività avente come scopo l'avanzamento di stato del prodotto; tali processi si dividono in:
        \begin{itemize}
            \item \textbf{Primari}: insiemi di attività che producono effetti direttamente sul \textit{software}, sulle sue caratteristiche e sulle relazioni con gli \textit{stakeholders};
            \item \textbf{Di supporto}: insiemi di attività che supportano i processi primari (ad esempio l'\textbf{automazione}, la \textbf{documentazione}, la \textbf{verifica} e la \textbf{validazione});
            \item \textbf{Organizzativi}: insiemi di attività atte alla gestione delle risorse umane (interne ed esterne) in termini di ruoli e interazioni tra essi ed alla gestione dei processi interni.
        \end{itemize}
\end{itemize}

%origini
\section{Origini}


%obiettivi

%principi di base

%framework agili

%ruoli

%sprint

%cerimonie

%artefatti