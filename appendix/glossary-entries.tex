% Acronyms


% \newacronym[description={\glslink{apig}{Application Program Interface}}]
%     {api}{API}{Application Program Interface}

% \newacronym[description={\glslink{umlg}{Unified Modeling Language}}]
%     {uml}{UML}{Unified Modeling Language}

\newacronym[description={\glslink{pwag}{\textit{Progressive Web App}}}]
    {pwa}{PWA}{\textit{Progressive Web App}}

\newacronym[description={\textit{Information Technology}, acronimo usato per indicare persone o cose attinenti all'ambito informatico}]
    {it}{IT}{\textit{Information Technology}}

\newacronym[description={\glslink{b2bg}{\textit{Business to business}}}]
    {b2b}{B2B}{\textit{Business to business}}

\newacronym[description={\glslink{wmsg}{\textit{Warehouse management system}}}]
    {wms}{WMS}{\textit{Warehouse management system}}

\newacronym[description={\glslink{damg}{\textit{Digital asset management}}}]
    {dam}{DAM}{\textit{Digital asset management}}

\newacronym[description={\glslink{mesg}{\textit{Manufacturing execution system}}}]
    {mes}{MES}{\textit{Manufacturing execution system}}

\newacronym[description={\glslink{erpg}{\textit{Enterprise resource planning}}}]
    {erp}{ERP}{\textit{Enterprise resource planning}}

% Glossary entries

% \newglossaryentry{apig} {
%     name=\glslink{api}{API},
%     text=Application Program Interface,
%     sort=api,
%     description={in informatica con il termine \emph{Application Programming Interface API} (ing. interfaccia di programmazione di un'applicazione) si indica ogni insieme di procedure disponibili al programmatore, di solito raggruppate a formare un set di strumenti specifici per l'espletamento di un determinato compito all'interno di un certo programma. La finalità è ottenere un'astrazione, di solito tra l'hardware e il programmatore o tra software a basso e quello ad alto livello semplificando così il lavoro di programmazione}
% }

% \newglossaryentry{umlg} {
%     name=\glslink{uml}{UML},
%     text=UML,
%     sort=uml,
%     description={in ingegneria del software \emph{UML, Unified Modeling Language} (ing. linguaggio di modellazione unificato) è un linguaggio di modellazione e specifica basato sul paradigma object-oriented. L'\emph{UML} svolge un'importantissima funzione di ``lingua franca'' nella comunità della progettazione e programmazione a oggetti. Gran parte della letteratura di settore usa tale linguaggio per descrivere soluzioni analitiche e progettuali in modo sintetico e comprensibile a un vasto pubblico}
% }

\newglossaryentry{pwag} {
    name=\glslink{pwa}{\textit{Progressive Web App}},
    text=\textit{Progressive Web App},
    sort=pwa,
    description={in informatica con il termine \textit{"Progressive Web App"} (it. applicazione web progressiva) si indica un'applicazione sviluppata utilizzando tecnologie utilizzate solitamente per lo sviluppo web, ma che offre un'esperienza utente simile a quella di un'app nativa:
    \begin{itemize}
        \item Come un sito web, può funzionare su piattaforme e dispositivi diversi utilizzando un unico codice sorgente;
        \item Come un'app specifica per una piattaforma, può essere installata sul dispositivo, può operare offline e in background, e può integrarsi con il dispositivo e con altre app installate.
    \end{itemize}
    \cite{site:pwa-mdn}}
}

\newglossaryentry{controllo-qualita}{
    name=Controllo qualità,
    text=Controllo qualità,
    sort=controllo-qualita,
    description={con il termine "controllo qualità" si riferisce alle fasi del sistema di gestione della qualità che prevedono ispezioni, test, esami e verifiche mirate a determinare il livello di soddisfacimento dei requisiti stabiliti per un determinato prodotto, servizio o processo}
}

\newglossaryentry{filiera-produttiva}{
    name=Filiera produttiva,
    text=Filiera produttiva,
    sort=filiera-produttiva,
    description={con il termine "filiera produttiva" si indica la sequenza delle lavorazioni (detta anche filiera tecnologico-produttiva), effettuate in successione, al fine di trasformare le materie prime in un prodotto finito (ingl. \textit{supply chain}). \\ 
    Le diverse imprese sono integrate tra loro (ai fini della realizzazione di un prodotto):
    \begin{itemize}
        \item \textbf{Verticalmente}: se svolgono una o più attività della filiera;
        \item \textbf{Orizzontalmente}: se operano allo stesso stadio di un ciclo produttivo.
    \end{itemize}
    Con la globalizzazione dell’economia, esse possono essere situate in paesi e continenti diversi.\\
    \cite{site:filiera-produttiva-treccani}}
}

\newglossaryentry{software-house}{
    name=\textit{Software house},
    text=\textit{Software house},
    sort=software-house,
    description={con il termine \textit{"software house"} si intende un’azienda specializzata nella produzione di software il cui obiettivo è quello di sviluppare applicazioni informatiche personalizzate per i propri clienti, che possano soddisfare le loro esigenze specifiche; si occupa dell’intero processo di sviluppo: dalle attività di analisi e progettazione, alla scrittura del codice, alla messa in produzione e manutenzione}
}

\newglossaryentry{digital-asset}{
    name=\textit{Digital asset},
    text=\textit{Digital asset},
    sort=digital-asset,
    description={con il termine \textit{"digital asset"} (it. risorsa digitale) si intende tutto ciò che esiste solo in forma digitale e viene fornito con un diritto di utilizzo distinto o un'autorizzazione in base all'uso. \\
    I dati che non possiedono tale diritto non sono considerati beni
    }
}

\newglossaryentry{tech-stack}{
    name=\textit{Stack} tecnologico,
    text=\textit{Stack} tecnologico,
    sort=stack,
    description={con il termine "\textit{stack} tecnologico" (it. pila di tecnologie) si intende l'insieme delle tecnologie utilizzate durante lo sviluppo, la manutenzione, il rilascio di un prodotto software; queste tecnologie possono essere i linguaggi di programmazione, i \textit{frameworks}, le librerie e, in generale, gli strumenti utilizzati nei processi citati.
    }
}

\newglossaryentry{b2bg}{
    name=\textit{Business to business},
    text=\textit{Business to business},
    sort=business,
    description={con il termine \textit{"business to business"} (it. commercio interaziendale) si intende la tipologia di commercio elettronico che intercorre tra attori economici organizzati in forma d’impresa, quali per esempio le aziende manifatturiere, industriali e commerciali, attraverso siti \textit{web} dedicati.
    }
}

\newglossaryentry{wmsg}{
    name=\glslink{wms}{\textit{Warehouse management system}},
    text=\textit{Warehouse management system},
    sort=warehouse,
    description={con il termine \textit{"warehouse management system"} (it. sistema di gestione del magazzino) si intende un \textit{software} che aiuta le aziende a gestire e controllare le operazioni quotidiane di magazzino, dall'ingresso delle merci e materiali in un centro di distribuzione o polo logistico fino alla loro uscita.
    }
}

\newglossaryentry{damg}{
    name=\glslink{dam}{\textit{Digital asset management}},
    text=\textit{Digital asset management},
    sort=digital-asset-management,
    description={con il termine \textit{"digital asset management"} (it. sistema di gestione delle risorse digitali) si intende un \textit{software} che consente di creare, organizzare e distribuire i contenuti su differenti canali e aumentare l’efficacia della comunicazione.\\ 
    È utilizzato per centralizzare e organizzare le risorse in un’unica libreria di facile accesso.
    }
}

\newglossaryentry{mesg}{
    name=\glslink{mes}{\textit{Manufacturing execution system}},
    text=\textit{Manufacturing execution system},
    sort=manufacturing,
    description={con il termine \textit{"manufacturing execution system"} (it. sistema di esecuzione manifatturiera) si intende un \textit{software} che ha la principale funzione di gestire e controllare la funzione produttiva di un'azienda. \\ La gestione riguarda il dispaccio degli ordini, gli avanzamenti in quantità e tempo, il versamento a magazzino, nonché il collegamento diretto ai macchinari per dedurre informazioni utili ad integrare l'esecuzione della produzione come a produrre informazioni per il controllo della produzione stessa.
    }
}

\newglossaryentry{erpg}{
    name=\glslink{erp}{\textit{Enterprise resource planning}},
    text=\textit{Enterprise resource planning},
    sort=enterprise,
    description={con il termine \textit{"enterprise resource planning"} (it. pianificazione delle risorse d'impresa) si intende un \textit{software} di gestione che integra tutti i processi aziendali e tutte le funzioni aziendali rilevanti, ad esempio vendite, acquisti, gestione magazzino, finanza o contabilità.
    }
}

\newglossaryentry{stakeholder}{
    name=\textit{Stakeholder},
    text=\textit{Stakeholder},
    sort=stakeholder,
    description={con il termine \textit{"stakeholder"} (it. portatore di interessi) si intendono tutti i soggetti, individui od organizzazioni, attivamente coinvolti in un’iniziativa economica (progetto, azienda), il cui interesse è negativamente o positivamente influenzato dal risultato dell’esecuzione, o dall’andamento, dell’iniziativa e la cui azione o reazione a sua volta influenza le fasi o il completamento di un progetto o il destino di un’organizzazione.
    }
}

\newglossaryentry{innovazione}{
    name=Innovazione,
    text=Innovazione,
    sort=innovazione,
    description={con il termine "innovazione" si intende l'atto e l'effetto dell'innovare, cioè dell'introdurre concetti, metodi, strumenti nuovi.
    }
}

\newglossaryentry{realta-aumentata}{
    name=Realtà aumentata,
    text=Realtà aumentata,
    sort=realta-aumentata,
    description= {con il termine "realtà aumentata" si intende la tecnica attraverso cui si aggiungono informazioni alla scena reale. \\ 
    Questa tecnica è realizzabile attraverso piccoli visori sostenuti, come i caschi immersivi, da supporti montati sulla testa che permettono di vedere la scena reale attraverso lo schermo semitrasparente del visore (\textit{see-through}), utilizzato anche per mostrare grafica e testi generati dal computer. }
}