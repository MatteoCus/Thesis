% Acronyms


% \newacronym[description={\glslink{apig}{Application Program Interface}}]
%     {api}{API}{Application Program Interface}

% \newacronym[description={\glslink{umlg}{Unified Modeling Language}}]
%     {uml}{UML}{Unified Modeling Language}

\newacronym[description={\glslink{pwag}{\textit{Progressive Web App}}}]
    {pwa}{PWA}{\textit{Progressive Web App}}

\newacronym[description={\textit{Information Technology}, acronimo usato per indicare persone o cose attinenti all'ambito informatico}]
    {it}{IT}{\textit{Information Technology}}

% Glossary entries

% \newglossaryentry{apig} {
%     name=\glslink{api}{API},
%     text=Application Program Interface,
%     sort=api,
%     description={in informatica con il termine \emph{Application Programming Interface API} (ing. interfaccia di programmazione di un'applicazione) si indica ogni insieme di procedure disponibili al programmatore, di solito raggruppate a formare un set di strumenti specifici per l'espletamento di un determinato compito all'interno di un certo programma. La finalità è ottenere un'astrazione, di solito tra l'hardware e il programmatore o tra software a basso e quello ad alto livello semplificando così il lavoro di programmazione}
% }

% \newglossaryentry{umlg} {
%     name=\glslink{uml}{UML},
%     text=UML,
%     sort=uml,
%     description={in ingegneria del software \emph{UML, Unified Modeling Language} (ing. linguaggio di modellazione unificato) è un linguaggio di modellazione e specifica basato sul paradigma object-oriented. L'\emph{UML} svolge un'importantissima funzione di ``lingua franca'' nella comunità della progettazione e programmazione a oggetti. Gran parte della letteratura di settore usa tale linguaggio per descrivere soluzioni analitiche e progettuali in modo sintetico e comprensibile a un vasto pubblico}
% }

\newglossaryentry{pwag} {
    name=\glslink{pwa}{\textit{Progressive Web App}},
    text=Progressive Web App,
    sort=pwa,
    description={in informatica con il termine \textit{"Progressive Web App"} (it. applicazione web progressiva) si indica un'applicazione sviluppata utilizzando tecnologie utilizzate solitamente per lo sviluppo web, ma che offre un'esperienza utente simile a quella di un'app nativa:
    \begin{itemize}
        \item Come un sito web, può funzionare su piattaforme e dispositivi diversi utilizzando un unico codice sorgente;
        \item Come un'app specifica per una piattaforma, può essere installata sul dispositivo, può operare offline e in background, e può integrarsi con il dispositivo e con altre app installate.
    \end{itemize}
    \cite{site:pwa-mdn}}
}

\newglossaryentry{controllo-qualita}{
    name=Controllo qualità,
    text=Controllo qualità,
    sort=controllo-qualita,
    description={con il termine "controllo qualità" si indica un processo metodico e sistematico implementato all'interno di un'organizzazione al fine di garantire che i prodotti o i servizi offerti rispettino degli standard di qualità stabiliti a priori. \\
    Questo processo coinvolge l'applicazione di tecniche, strumenti e protocolli specifici per valutare la conformità di un prodotto o servizio alle specifiche e ai requisiti prefissati}
}

\newglossaryentry{filiera-produttiva}{
    name=Filiera produttiva,
    text=Filiera produttiva,
    sort=filiera-produttiva,
    description={con il termine "filiera produttiva" si indica la sequenza delle lavorazioni (detta anche filiera tecnologico-produttiva), effettuate in successione, al fine di trasformare le materie prime in un prodotto finito (ingl. \textit{supply chain}). \\ 
    Le diverse imprese sono integrate tra loro (ai fini della realizzazione di un prodotto):
    \begin{itemize}
        \item \textbf{Verticalmente}: se svolgono una o più attività della filiera;
        \item \textbf{Orizzontalmente}: se operano allo stesso stadio di un ciclo produttivo.
    \end{itemize}
    Con la globalizzazione dell’economia, esse possono essere situate in paesi e continenti diversi.\\
    \cite{site:filiera-produttiva-treccani}}
}

\newglossaryentry{software-house}{
    name=Software house,
    text=\textit{Software house},
    sort=software-house,
    description={con il termine \textit{"software house"} si intende un’azienda specializzata nella produzione di software il cui obiettivo è quello di sviluppare applicazioni informatiche personalizzate per i propri clienti, che possano soddisfare le loro esigenze specifiche; si occupa dell’intero processo di sviluppo: dalle attività di analisi e progettazione, alla scrittura del codice, alla messa in produzione e manutenzione}
}

\newglossaryentry{digital-asset}{
    name=Software house,
    text=\textit{Digital asset},
    sort=digital-asset,
    description={con il termine \textit{"digital asset"} (it. risorsa digitale) si intende tutto ciò che esiste solo in forma digitale e viene fornito con un diritto di utilizzo distinto o un'autorizzazione in base all'uso. \\
    I dati che non possiedono tale diritto non sono considerati beni
    }
}

\newglossaryentry{tech-stack}{
    name=Stack tecnologico,
    text=\textit{Stack} tecnologico,
    sort=stack,
    description={con il termine "\textit{stack} tecnologico" (it. pila di tecnologie) si intende l'insieme delle tecnologie utilizzate durante lo sviluppo, la manutenzione, il rilascio di un prodotto software; queste tecnologie possono essere i linguaggi di programmazione, i \textit{frameworks}, le librerie e, in generale, gli strumenti utilizzati nei processi citati.
    }
}