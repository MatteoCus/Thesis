% \omiss produces '[...]'
\newcommand{\omissis}{[\dots\negthinspace]}

% Itemize symbols
% see: https://tex.stackexchange.com/a/62497
% \renewcommand{\labelitemi}{$\bullet$}
% \renewcommand{\labelitemii}{$\cdot$}
% \renewcommand{\labelitemiii}{$\diamond$}
% \renewcommand{\labelitemiv}{$\ast$}


\let\Chaptermark\chaptermark
% \Chaptername gives current chapter name
\def\chaptermark#1{\def\Chaptername{#1}\Chaptermark{#1}}
\makeatletter
% \currentname gives the current section name
\newcommand*{\currentname}{\@currentlabelname}
\makeatother

% Uncomment the following line for a different header/footer style
% \pagestyle{fancy} \setlength{\headheight}{14.5pt}
\fancyhead[L]{\fontsize{12}{14.5} \selectfont \thechapter. \Chaptername}
\fancyhead[R]{\fontsize{12}{14.5} \selectfont \currentname}
% Page number always in footer
\cfoot{\thepage}


% Custom hyphenation rules
\hyphenation {
    e-sem-pio
    ex-am-ple
}

% Images path
\graphicspath{{images/}}

% Page format settings
% see: http://wwwcdf.pd.infn.it/AppuntiLinux/a2547.htm
\setlength{\parindent}{14pt}    % first row indentation
\setlength{\parskip}{0pt}       % paragraphs spacing


% Load variables
\newcommand{\myName}{Cusin Matteo}
\newcommand{\myTitle}{ADeQA: una progressive web-app per il controllo qualità manuale in filiere produttive industriali}
\newcommand{\myDegree}{Tesi di laurea}
\newcommand{\myUni}{Università degli Studi di Padova}
\newcommand{\myFaculty}{Corso di Laurea in Informatica}
\newcommand{\myDepartment}{Dipartimento di Matematica ``Tullio Levi-Civita''}
\newcommand{\profTitle}{Prof.}
\newcommand{\myProf}{Vardanega Tullio}
\newcommand{\myLocation}{Padova}
\newcommand{\myAA}{2022/2023}
\newcommand{\myTime}{Dicembre 2023}

% PDF file metadata fields
% when updating them delete the build directory, otherwise they won't change
\begin{filecontents*}{\jobname.xmpdata}
  \Title{Tesi_Matteo_Cusin}
  \Author{Matteo Cusin}
  \Language{it-IT}
  \Subject{Relazione finale relativa all'applicazione ADeQA, sviluppata durante il periodo di tirocinio curricolare presso TriZeta}
  \Keywords{keyword1\sep keyword2\sep keyword3}
\end{filecontents*}


% Acronyms


% \newacronym[description={\glslink{apig}{Application Program Interface}}]
%     {api}{API}{Application Program Interface}

% \newacronym[description={\glslink{umlg}{Unified Modeling Language}}]
%     {uml}{UML}{Unified Modeling Language}

\newacronym[description={\glslink{pwag}{\textit{Progressive Web App}}}]
    {pwa}{PWA}{\textit{Progressive Web App}}

\newacronym[description={\textit{Information Technology}, acronimo usato per indicare persone o cose attinenti all'ambito informatico}]
    {it}{IT}{\textit{Information Technology}}

% Glossary entries

% \newglossaryentry{apig} {
%     name=\glslink{api}{API},
%     text=Application Program Interface,
%     sort=api,
%     description={in informatica con il termine \emph{Application Programming Interface API} (ing. interfaccia di programmazione di un'applicazione) si indica ogni insieme di procedure disponibili al programmatore, di solito raggruppate a formare un set di strumenti specifici per l'espletamento di un determinato compito all'interno di un certo programma. La finalità è ottenere un'astrazione, di solito tra l'hardware e il programmatore o tra software a basso e quello ad alto livello semplificando così il lavoro di programmazione}
% }

% \newglossaryentry{umlg} {
%     name=\glslink{uml}{UML},
%     text=UML,
%     sort=uml,
%     description={in ingegneria del software \emph{UML, Unified Modeling Language} (ing. linguaggio di modellazione unificato) è un linguaggio di modellazione e specifica basato sul paradigma object-oriented. L'\emph{UML} svolge un'importantissima funzione di ``lingua franca'' nella comunità della progettazione e programmazione a oggetti. Gran parte della letteratura di settore usa tale linguaggio per descrivere soluzioni analitiche e progettuali in modo sintetico e comprensibile a un vasto pubblico}
% }

\newglossaryentry{pwag} {
    name=\glslink{pwa}{\textit{Progressive Web App}},
    text=Progressive Web App,
    sort=pwa,
    description={in informatica con il termine \textit{"Progressive Web App"} (it. applicazione web progressiva) si indica un'applicazione sviluppata utilizzando tecnologie utilizzate solitamente per lo sviluppo web, ma che offre un'esperienza utente simile a quella di un'app nativa:
    \begin{itemize}
        \item Come un sito web, può funzionare su piattaforme e dispositivi diversi utilizzando un unico codice sorgente;
        \item Come un'app specifica per una piattaforma, può essere installata sul dispositivo, può operare offline e in background, e può integrarsi con il dispositivo e con altre app installate.
    \end{itemize}
    \cite{site:pwa-mdn}}
}

\newglossaryentry{controllo-qualita}{
    name=Controllo qualità,
    text=Controllo qualità,
    sort=controllo-qualita,
    description={con il termine "controllo qualità" si indica un processo metodico e sistematico implementato all'interno di un'organizzazione al fine di garantire che i prodotti o i servizi offerti rispettino degli standard di qualità stabiliti a priori. \\
    Questo processo coinvolge l'applicazione di tecniche, strumenti e protocolli specifici per valutare la conformità di un prodotto o servizio alle specifiche e ai requisiti prefissati}
}

\newglossaryentry{filiera-produttiva}{
    name=Filiera produttiva,
    text=Filiera produttiva,
    sort=filiera-produttiva,
    description={con il termine "filiera produttiva" si indica la sequenza delle lavorazioni (detta anche filiera tecnologico-produttiva), effettuate in successione, al fine di trasformare le materie prime in un prodotto finito (ingl. \textit{supply chain}). \\ 
    Le diverse imprese sono integrate tra loro (ai fini della realizzazione di un prodotto):
    \begin{itemize}
        \item \textbf{Verticalmente}: se svolgono una o più attività della filiera;
        \item \textbf{Orizzontalmente}: se operano allo stesso stadio di un ciclo produttivo.
    \end{itemize}
    Con la globalizzazione dell’economia, esse possono essere situate in paesi e continenti diversi.\\
    \cite{site:filiera-produttiva-treccani}}
}

\newglossaryentry{software-house}{
    name=Software house,
    text=\textit{Software house},
    sort=software-house,
    description={con il termine \textit{"software house"} si intende un’azienda specializzata nella produzione di software il cui obiettivo è quello di sviluppare applicazioni informatiche personalizzate per i propri clienti, che possano soddisfare le loro esigenze specifiche; si occupa dell’intero processo di sviluppo: dalle attività di analisi e progettazione, alla scrittura del codice, alla messa in produzione e manutenzione}
}

\newglossaryentry{digital-asset}{
    name=Software house,
    text=\textit{Digital asset},
    sort=digital-asset,
    description={con il termine \textit{"digital asset"} (it. risorsa digitale) si intende tutto ciò che esiste solo in forma digitale e viene fornito con un diritto di utilizzo distinto o un'autorizzazione in base all'uso. \\
    I dati che non possiedono tale diritto non sono considerati beni
    }
}

\newglossaryentry{tech-stack}{
    name=Stack tecnologico,
    text=\textit{Stack} tecnologico,
    sort=stack,
    description={con il termine "\textit{stack} tecnologico" (it. pila di tecnologie) si intende l'insieme delle tecnologie utilizzate durante lo sviluppo, la manutenzione, il rilascio di un prodotto software; queste tecnologie possono essere i linguaggi di programmazione, i \textit{frameworks}, le librerie e, in generale, gli strumenti utilizzati nei processi citati.
    }
}
\makeglossaries

\bibliography{appendix/bibliography}

\defbibheading{bibliography} {
    \cleardoublepage
    \phantomsection
    \addcontentsline{toc}{chapter}{\bibname}
    \chapter*{\bibname\markboth{\bibname}{\bibname}}
}

% Spacing between entries
\setlength\bibitemsep{1.5\itemsep}

\DeclareBibliographyCategory{opere}
\DeclareBibliographyCategory{web}

% \addtocategory{opere}{womak:lean-thinking}
\addtocategory{web}{site:pwa-mdn}
\addtocategory{web}{site:filiera-produttiva-treccani}

\defbibheading{opere}{\section*{Riferimenti bibliografici}}
\defbibheading{web}{\section*{Siti Web consultati}}

\hypersetup{
    %hyperfootnotes=false,
    %pdfpagelabels,
    colorlinks=true,
    linktocpage=true,
    pdfstartpage=1,
    pdfstartview=,
    breaklinks=true,
    pdfpagemode=UseNone,
    pageanchor=true,
    pdfpagemode=UseOutlines,
    plainpages=false,
    bookmarksnumbered,
    bookmarksopen=true,
    bookmarksopenlevel=1,
    hypertexnames=true,
    pdfhighlight=/O,
    %nesting=true,
    %frenchlinks,
    urlcolor=webbrown,
    linkcolor=RoyalBlue,
    citecolor=webgreen
    %pagecolor=RoyalBlue,
}

% Delete all links and show them in black
\if \isprintable 1
    \hypersetup{draft}
\fi

% Listings setup
\lstset{
    language=[LaTeX]Tex,%C++,
    keywordstyle=\color{RoyalBlue}, %\bfseries,
    basicstyle=\small\ttfamily,
    %identifierstyle=\color{NavyBlue},
    commentstyle=\color{Green}\ttfamily,
    stringstyle=\rmfamily,
    numbers=none, %left,%
    numberstyle=\scriptsize, %\tiny
    stepnumber=5,
    numbersep=8pt,
    showstringspaces=false,
    breaklines=true,
    frameround=ftff,
    frame=single
}

\definecolor{webgreen}{rgb}{0,.5,0}
\definecolor{webbrown}{rgb}{.6,0,0}

\newcommand{\sectionname}{sezione}
\addto\captionsitalian{\renewcommand{\figurename}{Figura}
                       \renewcommand{\tablename}{Tabella}}

\newcommand{\glsfirstoccur}{\ap{{[g]}}}

\newcommand{\intro}[1]{\emph{\textsf{#1}}}

% Risks environment
\newcounter{riskcounter}                % define a counter
\setcounter{riskcounter}{0}             % set the counter to some initial value

%%%% Parameters
% #1: Title
\newenvironment{risk}[1]{
    \refstepcounter{riskcounter}        % increment counter
    \par \noindent                      % start new paragraph
    \textbf{\arabic{riskcounter}. #1}   % display the title before the content of the environment is displayed
}{
    \par\medskip
}

\newcommand{\riskname}{Rischio}

\newcommand{\riskdescription}[1]{\textbf{\\Descrizione:} #1.}

\newcommand{\risksolution}[1]{\textbf{\\Soluzione:} #1.}

% Use case environment
\newcounter{usecasecounter}             % define a counter
\setcounter{usecasecounter}{0}          % set the counter to some initial value

%%%% Parameters
% #1: ID
% #2: Nome
\newenvironment{usecase}[2]{
    \renewcommand{\theusecasecounter}{\usecasename #1}  % this is where the display of
                                                        % the counter is overwritten/modified
    \refstepcounter{usecasecounter}             % increment counter
    \vspace{10pt}
    \par \noindent                              % start new paragraph
    {\large \textbf{\usecasename #1: #2}}       % display the title before the
                                                % content of the environment is displayed
    \medskip
}{
    \medskip
}

\newcommand{\usecasename}{UC}

\newcommand{\usecaseactors}[1]{\textbf{\\Attori Principali:} #1. \vspace{4pt}}
\newcommand{\usecasepre}[1]{\textbf{\\Precondizioni:} #1. \vspace{4pt}}
\newcommand{\usecasedesc}[1]{\textbf{\\Descrizione:} #1. \vspace{4pt}}
\newcommand{\usecasepost}[1]{\textbf{\\Postcondizioni:} #1. \vspace{4pt}}
\newcommand{\usecasealt}[1]{\textbf{\\Scenario Alternativo:} #1. \vspace{4pt}}

% Namespace description environment
\newenvironment{namespacedesc}{
    \vspace{10pt}
    \par \noindent  % start new paragraph
    \begin{description}
}{
    \end{description}
    \medskip
}

\newcommand{\classdesc}[2]{\item[\textbf{#1:}] #2}

\chaptertitlefont{\huge}
\chapternumberfont{\LARGE}

\makeatletter
\@removefromreset{footnote}{chapter}
\newcommand\footnoteref[1]{\protected@xdef\@thefnmark{\ref{#1}}\@footnotemark}
\makeatother