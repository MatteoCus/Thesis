\cleardoublepage
\phantomsection
\pdfbookmark{Sommario}{Sommario}
\begingroup
\let\clearpage\relax
\let\cleardoublepage\relax
\let\cleardoublepage\relax

\chapter*{Sommario}
L'elaborato descrive i processi, gli strumenti e le metodologie coinvolte nello sviluppo di una \textit{Progressive Web App}\footnote{\gls{pwag}}, ovvero di un'applicazione \textit{web} sviluppata per fornire un'esperienza simile a quella offerta da un'applicazione nativa, 
atta all'inserimento manuale di dati relativi al controllo qualità\footnote{\gls{controllo-qualita}} di filiere produttive\footnote{\gls{filiera-produttiva}} industriali. 
\newline
Nel dominio applicativo di interesse dell'elaborato:
\begin{itemize}
    \item \textbf{Controllo qualità}: è un processo atto a garantire che i prodotti / i servizi richiesti soddisfino degli \textit{standard} prefissati;
    \item \textbf{Filiera produttiva}: è la sequenza delle lavorazioni, effettuate in successione, aventi come fine la trasformazione delle materie prime in un prodotto finito (ingl. \textit{supply chain}).
\end{itemize} 
Il prodotto \textit{software}, sviluppato nel corso del tirocinio presso l'azienda \textit{\textbf{Trizeta S.r.l}} (d'ora in avanti \textit{\textbf{Trizeta}}) ha la peculiarità di doversi integrare in un \textit{software} già presente nella \textit{suite} aziendale e, al tempo stesso, essere in grado di eseguire in maniera del tutto indipendente replicando, all'occorrenza, alcune delle funzionalità presenti in esso.
%\vfill

\section*{Struttura del testo}
Il corpo principale della relazione è suddiviso in 4 capitoli:

\begin{description}
    \item[{\hyperref[cap:contesto-svolgimento]{Il primo capitolo}}] descrive il contesto in cui sono state svolte le attività di tirocinio curricolare, concludendo con una riflessione relativa al rapporto tra l'azienda ospitante e l'innovazione all'interno di processi e strumenti aziendali; 
    \item[{\hyperref[cap:motivazioni-tirocinio]{Il secondo capitolo}}] approfondisce le motivazioni che hanno consentito l'unione delle volontà del proponente e del sottoscritto al fine di acquisire nuove conoscenze e competenze (per il sottoscritto) e risolvere determinati bisogni relativi al dominio aziendale (per \textbf{\textit{Trizeta}});
    \item[{\hyperref[cap:elementi-progetto]{Il terzo capitolo}}] descrive i processi, gli strumenti e le modalità di esecuzione delle attività lavorative, oltre ai risultati conseguiti;
    \item[{\hyperref[cap:resoconto]{Il quarto capitolo}}] esegue una retrospettiva sul progetto, mettendo in relazione le competenze acquisite durante il percorso didattico e le competenze richieste dal tirocinio curricolare.
\end{description}

Di seguito all'ultimo capitolo, si trovano le sezioni:
\begin{itemize}
    \item \textbf{Acronimi e abbreviazioni}: ogni voce di questa sezione contiene un collegamento al relativo termine nel \textbf{Glossario};
    \item \textbf{Glossario}: contiene le definizioni dei termini specifici di dominio, con i collegamenti (se presenti) ai relativi acronimi o abbreviazioni; dopo ogni definizione, vengono resi disponibili dei collegamenti alle pagine in cui essi sono utilizzati;
    \item \textbf{Bibliografia}: in questa sezione specifico le fonti di informazione utilizzate per dare definizione ad un concetto, indicando eventuali collegamenti ipertestuali esterni, porzione di testo in cui sono state citate e termine al quale si riferiscono.
\end{itemize}

\section*{Convenzioni tipografiche}
Riguardo la stesura del testo, ho adottato le seguenti convenzioni tipografiche:
\begin{itemize}
	\item Gli acronimi, le abbreviazioni e i termini ambigui o di uso non comune menzionati vengono definiti nel capitolo \textbf{Glossario}:
        \begin{itemize}
            \item Al primo utilizzo di uno dei termini precedentemente indicati, verrà fornita un'essenziale definizione in sede di utilizzo;
            \item Il solo primo utilizzo di uno dei termini di cui sopra sarà accompagnato da una nota a piè di pagina contenente il riferimento al termine nel capitolo \textbf{Glossario}.
        \end{itemize} 
	\item Stile \textit{corsivo} per i termini in lingua straniera, nomi propri ed i termini facenti parte del gergo tecnico;
	\item Stile \textbf{grassetto} per il nome dell'azienda ospitante del periodo di tirocinio, i nomi dei capitoli del documento ed i termini chiave delle attività di tirocinio;
	\item Stile \textbf{grassetto} per le parole rilevanti in sezioni molto ampie di testo;
	\item Stile \textbf{grassetto} per i termini che definiscono da soli le voci di un elenco (sia esso numerato o puntato);
	\item Ogni voce di un elenco (puntato e numerato) si conclude con un punto e virgola ad eccezione dell'ultima voce, che terminerà con un punto;
	\item Colore rosso e \textit{font} \texttt{monospaziato} per ogni collegamento a pagine \textit{web};
	\item \textit{Font} \texttt{monospaziato} per gli elementi di codice sorgente;
	\item Le fonti delle immagini sono inserite come nota a piè di pagina, contenente il \textit{link} della pagina \textit{web} principale di appartenenza.
\end{itemize}

%\selectlanguage{english}
%\pdfbookmark{Abstract}{Abstract}
%\chapter*{Abstract}

%\selectlanguage{italian}

\endgroup

\vfill
