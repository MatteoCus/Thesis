\chapter{Retrospettiva delle attività}
\label{cap:resoconto}
Questo capitolo si occupa di fornire una valutazione retrospettiva sulle attività svolte ed il risultato ottenuto, mettendo a confronto:
\begin{itemize}
    \item Aspettive e risultati raggiunti;
    \item Competenze acquisite durante le attività e competenze erogate dal corso di studi.
\end{itemize}

\section{Soddisfacimento degli obiettivi prefissati}
% In questa sezione metterò in relazione gli obiettivi indicati in §2.5 ed i risultati indicati in §3.8.

\subsection{Obiettivi aziendali}

\subsubsection*{Obiettivi obbligatori}
\rowcolors{2}{white}{gray!25}
\begin{longtable}{>{\centering\arraybackslash}m{0.50\textwidth}>{\centering\arraybackslash}m{0.15\textwidth}>{\centering\arraybackslash}m{0.35\textwidth}}
    \hline
    \rowcolor{black}
    \color{white}\textbf{Obiettivo} & \color{white}\textbf{Soddisfatto} & \color{white}\textbf{Fonte} \\
    \hline
    \endhead % This line indicates the end of the header and the start of the repeated heading on subsequent pages
    \textbf{O01}: comprensione dei requisiti utente da soddisfare & SÌ & \hyperref[sec:analisi]{§3.3} Analisi, \hyperref[sec:validazione]{§3.7} Validazione \\
    \hline
    \textbf{O02}: studio dell’interfaccia dell’applicazione \textit{ADeMES}, che verrà integrata con il prodotto da sviluppare durante il tirocinio & SÌ & Tabella \hyperref[tab:colors]{3.7} \\
    \hline
    \textbf{O03}: acquisizione della sufficiente dimestichezza con i concetti di base del \textit{framework Angular} & SÌ & \hyperref[subsec:architettura]{§3.4.3} Progettazione - Architettura \\
    \hline
    \textbf{O04}: progettazione dell'interfaccia grafica in base allo stile dell'interfaccia dell'applicazione \textit{ADeMES} & SÌ & \hyperref[subsec:interfaccia]{§3.4.2} Progettazione - Interfaccia grafica  \\
    \hline
    \textbf{O05}: sviluppo di una versione di base dell'applicazione \textit{web} che consenta di eseguire le operazioni \textit{CRUD} sui dati di qualità & SÌ & \hyperref[sec:validazione]{§3.7} Validazione \\
    \hline
    \textbf{O06}: \textit{live demo} della web application in un ambiente simulato & SÌ & \hyperref[sec:validazione]{§3.7} Validazione \\
    \hline
    \textbf{O07}: studio e scelta (motivata) della tecnologia per la fruizione dell'applicazione in lingua inglese & SÌ & \hyperref[subsec:internazionalizzazione]{§3.4.1} Internazionalizzazione e localizzazione \\
    \hline
    \textbf{O08}: il \textit{software} deve potersi integrare nel software ADeMES mediante un elemento \texttt{<iframe>} \textit{HTML} & SÌ & \hyperref[subsec:integrazione]{§3.5} Codifica - Integrazione dell'applicazione con \textit{ADeMES}\\
    \hline
    \textbf{O09}: il \textit{software} deve poter essere eseguibile in modalità \textit{standalone} (in questo caso, in grado di funzionare anche senza l'ausilio del \textit{software ADeMES}) & SÌ & \hyperref[subsec:interfaccia-risultato]{§3.8.2}Risultato finale - interfaccia grafica \\
    \hline
    \caption{Obiettivi di tirocinio - obbligatori}
\end{longtable}


\section{Competenze e conoscenze acquisite}

In questa sezione descriverò le abilità e le conoscenze acquisite nel corso del tirocinio indicando (se necessario) i benefici ottenuti ed il loro grado di acquisizione.

\section{Competenze curricolari e lavorative}

In questa sezione discuterò della differenza tra le competenze acquisite ed erogate dal corso di studi e le competenze necessarie per lo svolgimento delle attività di tirocinio.
