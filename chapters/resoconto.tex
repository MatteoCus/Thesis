\chapter{Retrospettiva delle attività}
\label{cap:resoconto}
Questo capitolo si occupa di fornire una valutazione retrospettiva sulle attività svolte ed il risultato ottenuto, mettendo a confronto:
\begin{itemize}
    \item Aspettive e risultati raggiunti;
    \item Competenze acquisite durante le attività e competenze erogate dal corso di studi.
\end{itemize}

\section{Soddisfazione degli obiettivi prefissati}
% In questa sezione metterò in relazione gli obiettivi indicati in §2.5 ed i risultati indicati in §3.8.

\subsection{Obiettivi aziendali}
Gli obiettivi che riporto di seguito fanno riferimento alla sezione \hyperref[sec:obiettivi-aziendali]{§2.3}.
\subsubsection*{Obiettivi obbligatori}
\rowcolors{2}{white}{gray!25}
\begin{longtable}{>{\centering\arraybackslash}m{0.50\textwidth}>{\centering\arraybackslash}m{0.15\textwidth}>{\centering\arraybackslash}m{0.35\textwidth}}
    \hline
    \rowcolor{black}
    \color{white}\textbf{Obiettivo} & \color{white}\textbf{Soddisfatto} & \color{white}\textbf{Fonte} \\
    \hline
    \endhead % This line indicates the end of the header and the start of the repeated heading on subsequent pages
    \textbf{O01}: comprensione dei requisiti utente da soddisfare & SÌ & \hyperref[sec:analisi]{§3.3} Analisi, \hyperref[sec:validazione]{§3.7} Validazione \\
    \hline
    \textbf{O02}: studio dell’interfaccia dell’applicazione \textit{ADeMES}, che verrà integrata con il prodotto da sviluppare durante il tirocinio & SÌ & Tabella \hyperref[tab:colors]{3.7} \\
    \hline
    \textbf{O03}: acquisizione della sufficiente dimestichezza con i concetti di base del \textit{framework Angular} & SÌ & \hyperref[subsec:architettura]{§3.4.3} Progettazione - Architettura \\
    \hline
    \textbf{O04}: progettazione dell'interfaccia grafica in base allo stile dell'interfaccia dell'applicazione \textit{ADeMES} & SÌ & \hyperref[subsec:interfaccia]{§3.4.2} Progettazione - Interfaccia grafica  \\
    \hline
    \textbf{O05}: sviluppo di una versione di base dell'applicazione \textit{web} che consenta di eseguire le operazioni \textit{CRUD} sui dati di qualità & SÌ & \hyperref[sec:validazione]{§3.7} Validazione \\
    \hline
    \textbf{O06}: \textit{live demo} della web application in un ambiente simulato & SÌ & \hyperref[sec:validazione]{§3.7} Validazione \\
    \hline
    \textbf{O07}: studio e scelta (motivata) della tecnologia per la fruizione dell'applicazione in lingua inglese & SÌ & \hyperref[subsec:internazionalizzazione]{§3.4.1} Internazionalizzazione e localizzazione \\
    \hline
    \textbf{O08}: il \textit{software} deve potersi integrare nel software ADeMES mediante un elemento \texttt{<iframe>} \textit{HTML} & SÌ & \hyperref[subsec:integrazione]{§3.5} Codifica - Integrazione dell'applicazione con \textit{ADeMES}\\
    \hline
    \textbf{O09}: il \textit{software} deve poter essere eseguibile in modalità \textit{standalone} (in questo caso, in grado di funzionare anche senza l'ausilio del \textit{software ADeMES}) & SÌ & \hyperref[subsec:interfaccia-risultato]{§3.8.2}Risultato finale - interfaccia grafica \\
    \hline
    \caption{Obiettivi di tirocinio - obbligatori}
\end{longtable}

\subsubsection*{Obiettivi desiderabili}
\rowcolors{2}{white}{gray!25}
\begin{longtable}{>{\centering\arraybackslash}m{0.50\textwidth}>{\centering\arraybackslash}m{0.15\textwidth}>{\centering\arraybackslash}m{0.35\textwidth}}
    \hline
    \rowcolor{black}
    \color{white}\textbf{Obiettivo} & \color{white}\textbf{Soddisfatto} & \color{white}\textbf{Fonte} \\
    \hline
    \endhead % This line indicates the end of the header and the start of the repeated heading on subsequent pages
    \textbf{D01}: ottimizzazione dei servizi esposti & NO & - \\
    \hline
    \caption{Obiettivi di tirocinio - desiderabili}
\end{longtable}
L'obiettivo \textbf{D01} non è stato soddisfatto a causa del poco tempo rimanente al termine delle attività di sviluppo del \textit{software ADeQA}.

\subsubsection*{Obiettivi facoltativi}
\rowcolors{2}{white}{gray!25}
\begin{longtable}{>{\centering\arraybackslash}m{0.50\textwidth}>{\centering\arraybackslash}m{0.15\textwidth}>{\centering\arraybackslash}m{0.35\textwidth}}
    \hline
    \rowcolor{black}
    \color{white}\textbf{Obiettivo} & \color{white}\textbf{Soddisfatto} & \color{white}\textbf{Fonte} \\
    \hline
    \endhead % This line indicates the end of the header and the start of the repeated heading on subsequent pages
    \textbf{F01}: ottimizzazione dell'esperienza utente per compatibilità con \textit{ADeMES} & SÌ & \hyperref[sec:validazione]{§3.7} Validazione \\
    \hline
    \textbf{F02}: ottimizzazione dell'interfaccia grafica, per rendere quanto più simile il prodotto a \textit{ADeMES} & SÌ & Tabella \hyperref[tab:colors]{3.7} \\
    \hline
    \textbf{F03}: possibilità di fruizione dell'applicazione in lingua spagnola & SÌ & \hyperref[subsec:internazionalizzazione]{§3.4.1} Internazionalizzazione e localizzazione \\
    \hline
    \caption{Obiettivi di tirocinio - facoltativi}
\end{longtable}

\subsection{Obiettivi personali}
Gli obiettivi che riporto di seguito fanno riferimento alla sezione \hyperref[sec:obiettivi-personali]{§2.6}.

\rowcolors{2}{white}{gray!25}
\begin{longtable}{>{\centering\arraybackslash}m{0.50\textwidth}>{\centering\arraybackslash}m{0.15\textwidth}>{\centering\arraybackslash}m{0.35\textwidth}}
    \hline
    \rowcolor{black}
    \color{white}\textbf{Obiettivo} & \color{white}\textbf{Soddisfatto} & \color{white}\textbf{Fonte} \\
    \hline
    \endhead % This line indicates the end of the header and the start of the repeated heading on subsequent pages
    Capire come convertire una \textit{web application} in una \textit{Progressive Web App} & SÌ & \hyperref[sec:validazione]{§3.7} Validazione \\
    \hline
    Sviluppare un'interfaccia grafica che si adatti a \textit{desktop}, \textit{tablet} e \textit{smartphone} & SÌ & \hyperref[subsec:interfaccia]{§3.4.2} Progettazione - Interfaccia grafica \\
    \hline
    Comprendere come rendere fruibile in più lingue un prodotto \textit{software} & SÌ & \hyperref[subsec:internazionalizzazione]{§3.4.1} Internazionalizzazione e localizzazione \\
    \hline
    Capire come si possono gestire diversi temi grafici (tipicamente identificati come "tema chiaro" e "tema scuro") in un'interfaccia grafica \textit{web} \textit{software} & SÌ & \hyperref[subsec:interfaccia-risultato]{§3.8.2}Risultato finale - interfaccia grafica \\
    \hline
    Ideare un prodotto in grado di integrarsi con successo in un \textit{software} già esistente & SÌ & \hyperref[sec:validazione]{§3.7} Validazione \\
    \hline
    \caption{Obiettivi di tirocinio - personali}
\end{longtable}

\subsection{Commento}
Ho conseguito in modo soddisfacente quasi tutti gli obiettivi nel corso delle attività di tirocinio, in particolare tutti gli obiettivi obbligatori aziendali e gli obiettivi personali: valuto positivamente l'esperienza di \textit{stage} sia per quanto concerne il risultato raggiunto, 
sia per le modalità ed il clima di svolgimento delle attività.

\section{Competenze e conoscenze acquisite}

In questa sezione descriverò le abilità e le conoscenze acquisite nel corso del tirocinio indicando (se necessario) i benefici ottenuti ed il loro grado di acquisizione.

\section{Competenze curricolari e lavorative}

In questa sezione discuterò della differenza tra le competenze acquisite ed erogate dal corso di studi e le competenze necessarie per lo svolgimento delle attività di tirocinio.
