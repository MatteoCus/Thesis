\chapter{Retrospettiva delle attività}
\label{cap:resoconto}
Questo capitolo si occupa di fornire una valutazione retrospettiva sulle attività svolte ed il risultato ottenuto, mettendo a confronto:
\begin{itemize}
    \item Aspettive e risultati raggiunti;
    \item Competenze acquisite durante le attività e competenze erogate dal corso di studi.
\end{itemize}

\section{Soddisfacimento degli obiettivi prefissati}
% In questa sezione metterò in relazione gli obiettivi indicati in §2.5 ed i risultati indicati in §3.8.

\subsection*{Obiettivi aziendali}

\rowcolors{2}{gray!25}{white}
\begin{longtable}{>{\centering\arraybackslash}m{0.50\textwidth}>{\centering\arraybackslash}m{0.15\textwidth}>{\centering\arraybackslash}m{0.35\textwidth}}
    \hline
    \rowcolor{black}
    \color{white}\textbf{Obiettivo} & \color{white}\textbf{Soddisfatto} & \color{white}\textbf{Fonte} \\
    \hline
    \endhead % This line indicates the end of the header and the start of the repeated heading on subsequent pages
    \textbf{O01}: comprensione dei requisiti utente da soddisfare & SÌ & \hyperref[sec:analisi]{§3.3} Analisi, \hyperref[sec:validazione]{§3.7} Validazione \\
    \hline
    \textbf{O02}: studio dell’interfaccia dell’applicazione \textit{ADeMES}, che verrà integrata
    con il prodotto da sviluppare durante il tirocinio & SÌ & \hyperref[sec:analisi]{§3.4.2} Progettazione - Interfaccia grafica \\
    \hline
    \caption{Requisiti di qualità}
\end{longtable}

\section{Competenze e conoscenze acquisite}

In questa sezione descriverò le abilità e le conoscenze acquisite nel corso del tirocinio indicando (se necessario) i benefici ottenuti ed il loro grado di acquisizione.

\section{Competenze curricolari e lavorative}

In questa sezione discuterò della differenza tra le competenze acquisite ed erogate dal corso di studi e le competenze necessarie per lo svolgimento delle attività di tirocinio.
