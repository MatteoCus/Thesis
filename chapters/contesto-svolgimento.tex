\chapter{Contesto di svolgimento delle attività}
\label{cap:contesto-svolgimento}

Questo capitolo si occupa di fornire informazioni in merito a \textit{\textbf{Trizeta}}, azienda ospitante del tirocinio, al settore in cui essa opera (quindi anche ai beni e servizi offerti), 
al suo rapporto con l'introduzione di novità / con il miglioramento di processi e strumenti già in uso ed ai processi in essa utilizzati. \\
Le informazioni riportate di seguito sono frutto di osservazioni personali e dialoghi avuti nel corso del tirocinio.

\section{Introduzione all'azienda ospitante}

% In questa sezione descriverò brevemente l'azienda, indicandone dimensioni, ubicazione, figure presenti all'interno dell'organico, gruppo commerciale di cui fa parte.
\textit{\textbf{Trizeta}} è una \textit{software house}\footnote{\gls{software-house}}, ovvero un'azienda che si occupa dello sviluppo e della commercializzazione di \textit{software}, specializzata nella consulenza e nello sviluppo
di prodotti per aziende che desiderano l'automazione (totale o parziale) delle proprie attività industriali (compresa la gestione del magazzino); essa consente inoltre alle aziende clienti di gestire le proprie risorse digitali multimediali (i cosiddetti \textit{digital assets}\footnote{\gls{digital-asset}}). \\
L'azienda è ubicata a \textit{Monselice (Padova)} e dispone all'incirca di una decina di dipendenti \textit{IT}\footnote{\gls{it}} (informatici) tra loro eterogenei per anni di esperienza nel settore informatico, età anagrafica, e 
\textit{stack} tecnologico\footnote{\gls{tech-stack}} abitualmente utilizzato (tecnologie utilizzate e ambito di utilizzo delle stesse). \\
Recentemente \textit{\textbf{Trizeta}} è entrata a far parte di \textit{SYS-DAT Group}: è un gruppo di aziende specializzate nello sviluppo e manutenzione di prodotti \textit{software} rivolti ad aziende appartenenti a vari settori quali 
il settore moda (settore di origine di \textit{SYS-DAT}, azienda fondatrice del gruppo) ed il settore alimentare. 


\section{Prodotti e servizi}

% In questa sezione descriverò brevemente i prodotti ed i servizi offerti dall'azienda che ho potuto vedere (mi focalizzerò su un prodotto in particolare, dato che il software che ho prodotto nel corso del tirocinio andrà ad essere integrato con esso).
% Terminerò la sezione indicando la clientela a cui l'azienda si rivolge.
Come già indicato nella sezione precedente, \textit{\textbf{Trizeta}} intrattiene relazioni commerciali esclusivamente di tipo \textit{B2B}\footnote{\gls{b2bg}}: questa visione si riflette inevitabilmente sui prodotti offerti
e sull'insieme dei requisiti utente soddisfatti dai prodotti di seguito elencati.

\begin{itemize}
    \item \textit{ADeWMS}: è un \textit{WMS}\footnote{\gls{wmsg}} (gestionale relativo al contenuto e alle attività di magazzino) in grado di integrarsi con software \textit{ERP}\footnote{\gls{erpg}} e gestire ordini commerciali, consegne e relativa documentazione;
    \item \textit{P4NDOR4}: è un \textit{DAM}\footnote{\gls{damg}} (gestionale per \glslink{digital-asset}{\textit{digital assets}} aziendali) con possibilità di richiedere delle risorse direttamente a \textit{\textbf{Trizeta}};
    \item \textit{ADeMES}
\end{itemize}

\section{Processi interni}

In questa sezione descriverò ciò che ho potuto osservare nel periodo di tirocinio relativamente alle attività lavorative degli altri dipendenti IT, descrivendo il supporto ricevuto in relazione al mio tirocinio.


\section{Rapporto con l'innovazione}

In questa sezione descriverò quello che, in base alle mie osservazioni e all'esperienza dovuta alle attività di tirocinio, è il rapporto dell'azienda con tutto ciò che riguarda il miglioramento di processi / attività / prodotti già esistenti o l'introduzione di novità.