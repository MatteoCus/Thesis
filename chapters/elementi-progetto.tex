\chapter{Elementi caratterizzanti del progetto}
\label{cap:elementi-progetto}
Qui introdurrò brevemente il contenuto delle sezioni sottostanti.

\section{Stile lavorativo}

In questa sezione descriverò il modo in cui ho lavorato nel corso del tirocinio, descrivendo attività esterne allo sviluppo, alla verifica ed alla validazione e descrivendo il modo in cui invece tali attività sono state ideate.

\section{Strumenti utilizzati}

In questa sezione descriverò gli strumenti utilizzati nel corso del progetto, suddividendoli in base alle attività in cui essi sono stati impiegati.

\section{Analisi dei requisiti utente}

In questa sezione descriverò lo scopo dell'analisi dei requisiti in un progetto, le problematiche riscontrate e mostrerò i principali casi d'uso ed i principali requisiti elaborati.

\section{Progettazione}

In questa sezione descriverò lo scopo della progettazione in un progetto, le problematiche riscontrate e mostrerò le classi (e le loro dipendenze, quando utile e possibile, tramite il linguaggio UML) relative ai principali requisiti analizzati nella sezione precedente in modo da dare riscontro effettivo del passaggio da "requisito" a "scelta progettuale".
In questa sezione includerò anche la progettazione dell'interfaccia grafica relativa alle classi sopra indicate.

\section{Codifica}

In questa sezione descriverò lo scopo della codifica in un progetto ed indicherò le problematiche riscontrate.

\section{Verifica}

In questa sezione descriverò le modalità con le quali si è accertato che l’esecuzione delle attività (per un determinato periodo di tempo) non abbia introdotto errori.

\section{Validazione}

In questa sezione descriverò le modalità con le quali si è accertato che il prodotto finito fosse conforme alle aspettative.

\section{Statistiche qualitative e quantitative finali}

In questa sezione descriverò i prodotti di progetto dal punto di vista della qualità (qui vi sarà particolare enfasi sul prodotto software) e della quantità.